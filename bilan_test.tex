\documentclass{beamer}
\usepackage[T1]{fontenc}
\usepackage[utf8]{inputenc}
\usepackage{lmodern}
\usepackage[francais]{babel}
\usepackage{graphicx}
\usepackage{beamerthemeWarsaw}
\expandafter\def\expandafter\insertshorttitle\expandafter{\insertshorttitle\hfill\insertframenumber\,/\,\inserttotalframenumber}

\title{Projets Test - Groupe 6}
\author{Kevin \bsc{Boulala}, Maxime \bsc{Dubois}}
\institute{Université de Franche Comté}
\date{\today}

\begin{document}

  \begin{frame}
    \titlepage
  \end{frame}

  \begin{frame}
    \setcounter{tocdepth}{2}
	  \tableofcontents[]
  \end{frame}
  
  \section{Développeur}
  
    \begin{frame}
    	\tableofcontents[currentsection]
    \end{frame}
  
    \subsection{Présentation de Mark Attacks}
      \begin{frame}
        \frametitle{Mark Attacks}
        \begin{block}{}
          \begin{itemize}
            \item Pour les enseignants et les étudiants
            \item Saisir les notes et les afficher
            \item Gestion des modules
            \item Mais aussi des statistiques diverses
          \end{itemize}
        \end{block}
      \end{frame}
    
    \subsection{Les fonctionnalités développées}
      \begin{frame}
        \frametitle{Les fonctionnalités développées}
        \begin{block}{}
          \begin{itemize}
            \item Gestion des enseignants - Ajout
            \item Gestion des étudiants - Ajout
            \item Gestion des UE - Ajout
            \item Gestion des UE - Suppresion des modules
            \item Connexion de l'administrateur, d'un étudiant, d'un enseignant
            \item Validation : login, mot de passe, prénom, nom, adresse, date de naissance, mail, téléphone, description, code module, titre module
            \item Site résistant aux attaques
            \item Navigation du site (sur divers navigateurs)
            \item Navigation mobile
          \end{itemize}
        \end{block}
      
      \end{frame}
    
    \subsection{Les fonctionnalités non développées}
      \begin{frame}
        \frametitle{Les fonctionnalités non développées 1/4}
        \begin{block}{}
          \begin{itemize}
            \item Gestion des enseignants - Recherche
            \item Gestion des enseignants - Approbation compte
            \item Gestion des enseignants - Suppression
            \item Gestion des enseignants - Modification
            \item Gestion des étudiants - Modification
            \item Gestion des etudiants - Recherche
            \item Gestion des etudiants - Approbation compte
            \item Gestion des etudiants - Suppression
            \item Gestion des UE - Modifier module
            \item Gestion des modules - Recherche
            \item Gestion des modules - Affectation
          \end{itemize}
        \end{block}
      \end{frame}
      \begin{frame}
        \frametitle{Les fonctionnalités non développées 2/4}
        \begin{block}{}
          \begin{itemize}
            \item Gestion des enseignants - Approbation module
            \item Ajout d'autres enseignants
            \item Recherche d'étudiants
            \item Accès des informations
            \item Ajout d'étudiants
            \item Suppression d'enseignants
            \item Gestion des notes
            \item Modification des notes
            \item Ajout du coefficient du module
            \item Modification du coefficiant du module
            \item Suppression des notes
          \end{itemize}
        \end{block}
      \end{frame}
      \begin{frame}
        \frametitle{Les fonctionnalités non développées 3/4}
        \begin{block}{}
          \begin{itemize}
            \item Reset password
            \item Demande de compte enseignant, étudiant
            \item Espace utilisateur - Connexion
            \item Espace utilisateur - Consulation des stats recapitulatives
            \item Espace utilisateur - Consulation des notes
            \item Espace utilisateur - Consulation des stats
            \item Description compte utilisateur
            \item Changement des informations personnelles
            \item Description d'un module
            \item Notifications
            \item Liste des référents
          \end{itemize}
        \end{block}
      \end{frame}
      \begin{frame}
        \frametitle{Les fonctionnalités non développées 4/4}
        \begin{block}{}
          \begin{itemize}
            \item Notifications administrateur, enseignant, étudiant
          \end{itemize}
          \begin{center}
            Voilà.
          \end{center}
        \end{block}
      \end{frame}

    \subsection{Bilan des fonctionnalités}
    \begin{frame}
      \frametitle{Bilan à propos des fonctionnalités}
      \begin{block}{Les fonctionnalités développées}
        \begin{center}
          21 exigences développées
        \end{center}
      \end{block}
      \begin{block}{Les fonctionnalités non-développées}
        \begin{center}
          35 exigences non-développées
        \end{center}
      \end{block}
    \end{frame}

    \subsection{Démonstration}
      \begin{frame}
        \frametitle{Démonstration}
	      \begin{center}
          {\LARGE Lançons \href{http://localhost/m2test6/markattaks-tmp/website/}{Mark Attacks}}
	      \end{center}
	    \end{frame}
    
    \subsection{Synthèse du travail réalisé}
    
      \subsubsection{Infrastructure}
      
      \subsubsection{Tests d'acceptation}
      
      \subsubsection{Tests d'intégration}
      
      \subsubsection{Tests unitaires}
      
      \subsubsection{Bilan des problèmes et de leurs solutions}
      
    \subsection{Synthèse du travail en équipe}
      
      \subsubsection{Organisation de l'équipe}
      
      \subsubsection{Les évolutions au cours du projet}
      
      \subsubsection{Les problèmes ou améliorations possibles en fin de projet ?}
      
      \subsubsection{Le jour de la marmotte}
      % si on pouvait recommencer le projet, qu'est ce qu'on garderait, qu'est ce qu'on changerait
      
    \subsection{Bilan développeur}
    
  \section{Client}
  
    \begin{frame}
    	\tableofcontents[currentsection]
    \end{frame}
  
    \subsection{Conformité du produit par rapport à l'attendu}
    
    \subsection{Synthèse du statut d'acceptation des exigences}
    
    \subsection{Bilan client}
      

\end{document}
